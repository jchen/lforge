% !TEX root = ../paper.tex
\section{Discussion}
\subsection{Contributions}

We summarize below some of the main contributions we make:

\begin{description}
    \item[\textsc{Lforge} as a tool] First and foremost, we've created a tool that contributes utility both to Forge and to Lean. By allowing Forge specifications to be ported seamlessly into the Lean theorem prover, users are empowered to complement Forge's automated search capabilities with writing proofs for conjectured properties regarding their model. This creates a `hypothesize-and-proof' workflow for users and students to specify properties about a model specification, quickly prototype and test the validity of their models on small bounded examples, and delve deeper to proving said properties more generally. By preserving Forge's semantics, we've allowed for Forge model specifications to be `ported' into Lean, and users can feel comfortable that they are indeed modeling within the same relational framework that they are used to. 
    \item[Usability-first translation] \textsc{Lforge} is guided by usability and simplicity. Translations follow a simplest-first approach (see \cref{sec:join-cross-subset}) that focuses on implementations tailored to specific type configurations over the most general translation. Compromises are made to create a \emph{subset} of Forge that is most easily expressible in Lean, and users are guided to make changes and annotations that aid their translation. By leveraging Lean's LSP capabilities (see \cref{sec:dsl}), we can also expose the most relevant type and error information for our end-user. 
    \item[A Lean DSL] \textsc{Lforge} also serves as one of few examples of programs that utilize Lean's metaprogramming capabilities to implement a domain-specific language within Lean itself. We test Lean's exposed metaprogramming capabilities to their limits (see \cref{sec:dsl}), from type unification/checking, error reporting, on-hover hinting, as well as mixed specification, and produce an embedding of Forge that interacts easily with its host language. 
\end{description}

\subsection{Future Work}\label{sec:future-work}

We recognize that there is much work that is left to be done in this project. On top of the administrative work that remains---creating a coherent set of documentation and examples for \textsc{Lforge} and preparing it for general use---we detail below some of the major areas that are yet to be explored. 

\subsubsection{Formal Guarantees}

Throughout this implementation, we reference to the want for our translation to be \emph{faithful}, that is, we optimally want some form of soundness guarantee. We surely do not want to be able to prove a property about a Forge specification in Lean that doesn't hold (that is, Forge can find a counterexample). While we acknowledge completeness is not possible\footnote{Forge specifications are decidable because they compile to \textsc{sat}, which is decidable. \textsc{Lforge} translates properties into formulas in first-order logic, for which satisfiability is undecidable. One could imagine specifying a Turing machine and attempting to prove a predicate that it halts in a finite number of steps.}, we still hope for model properties to be generally provable---that is, it would be conceited if we were unable to prove any \textsc{Lforge}-translated specification. 

Both Forge and Lean, existing atop sound logical frameworks, can be formalized as logical objects. We ought to be able to prove properties about said translation in this higher logical framework, which should give us additional confidence in our translations. 

\subsubsection{A Proper Type System for Forge}

While we discuss the merits of inheriting Lean's type system in \cref{sec:dsl}, this process was without friction. Forge's forgiving type system is the source of a lot of conscious design choices (see \cref{sec:everything-is-a-set,sec:join-cross-subset}) as well as workarounds (see \cref{sec:type-coercions}). In its current state, Lean's elaborator (see \cref{sec:elaboration}), which contains the type unification algorithm, is not able to fully resolve types emitted out of Forge models due to the number of alternatives and implementations that are type-dependent. 

A proper, albeit tedious, solution to this issue is to intervene before elaboration to do a specific first-pass type check and type inference that is specific to Forge. This allows us to, at translation-time, include more type hints and type annotations for Lean's elaborator, eliminating the need to manually annotate types when Lean cannot infer coercions, as in \cref{sec:type-coercions}. This type system will be tailored to the complex behaviors of Forge types, and reduce the number of metavariables emitted as a result of our translation, which is currently a main source of confusion for the Lean elaborator\footnote{In other words, there is currently \emph{not enough} type information that comes out of our translation for Lean to fully figure out types, especially given the complex coercion structures and type class structures we've built out to support Forge operations.}. 

\subsubsection{Toward a Comprehensive Proof \& Tactic System}

We discuss in \cref{sec:mutex} some of the friction that our translation introduces, particularly around sets (see \cref{sec:everything-is-a-set}) and cardinality. While we can make use of techniques such as annotating instances and axioms with the \texttt{simp} modifier, proofs with translated Forge specifications still lack behind proofs in Lean. 

Mathematical proofs in Lean have the backing of \texttt{mathlib4} \cite{moura2021lean}, which contain a large library of theorems pertaining to mathematical objects they describe that complement the proof process. Translated \textsc{Lforge} specifications do not have such a foundation to build on. Especially where it pertains to data structures that are already sparsely supported (such as \texttt{Set}s, transitive closures, etc.) or custom implementations that we implement ourselves (such as those in \cref{sec:join-cross-subset}, like relational join or cross-products), many proofs are excessively cumbersome. 

Up until now, we have largely combatted this issue by modifying our \emph{translation} to be more granular, specific, and simpler when possible, such that emitted translations can be as simplified as possible. Future work should turn away from this solution in favor of a comprehensive library of theorems that are generally applicable to proving facts about the style of formulas generated by Forge specifications. Specifically, we need to work on developing a library of theorems and proof tactics that cater to Forge's `set-styled' statements. Only then will we be able to use Lean to its fullest extent complementing the automated search capabilities of Forge. 

\subsubsection{\textsc{Lforge} as a Pedagogical Tool}

As mentioned in \cref{sec:bg-forge}, Forge is a language with pedagogy in mind. We've also designed a tool that focuses on usability and learnability (see \cref{sec:dsl}) with a context and background that centers around pedagogical formal methods (see \cref{sec:toy-example}). We are interested in exploring this side of \textsc{Lforge}---that it can be used as a tool for students of different formal methods courses to bridge their learning and work on a meaningful and significant modeling project: using Forge to prototype and `check' properties automatically and formalizing their properties using proofs in Lean. This could contribute to a more complete and comprehensive formal methods workflow for students to explore more complex and interesting problems. 

\subsection{Lessons Learnt}

We close with some lessons learned and knowledge gained from this project. 

We echo the sentiment in \cite{gladshtein2024small} that documentation surrounding Lean's metaprogramming capabilities is still in its beta stages. Without expertise and knowledge on the inner workings of Lean which were largely undocumented, this project would have proved to be more challenging. Code search\footnote{Of publicly available GitHub repositories.}, browsing the Lean Community Zulip, and trial-and-error were essential in much of the progress made. 

\emph{Fairy tales do not always have happy endings.} We set out to port a significant subset of Forge into Lean, and the goal was for most existing Forge specifications to interoperate with Lean out-of-the-box. As we saw in \cref{sec:implementation} (especially \cref{sec:everything-is-a-set,sec:type-coercions}), and even in our example \cref{sec:mutex}, the difficulties were plenty. Retrofitting a type system that already exists---Lean's---onto a largely untyped specification model is difficult! 

However, we can make compromises (elegantly), as we did in \cref{sec:forge-model} excluding certain quantifiers that proved difficult to model, or introducing additional syntax as in \cref{sec:type-coercions} to make the task of translation easier for us. Forge already has language levels that suit different levels of learning and understanding \cite{ngpdbccdlrrvwwk-oopsla-2024}. With \textsc{Lforge}, Forge has gained another sublanguage that is most suited for the two-sided task of automated verification as well as manual formal verification via proofs that keep usability in mind. 
