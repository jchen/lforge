% !TEX root = ../paper.tex

\section*{Appendices}
\addcontentsline{toc}{section}{Appendices}
\renewcommand{\thesubsection}{\Alph{subsection}}

\subsection[]{Data Availability}

The source code for \textsc{Lforge}, including all mentioned examples and proofs, is available at this repository: \url{https://github.com/jchen/lforge}. The source code at the time of this thesis being submitted is tagged \texttt{thesis}. \todo{Check tag.}\textsc{Lforge} is available as a package and can be included as a dependency using Lake. 

\subsection[]{Barber Paradox Proof}\label{appendix:barber-proof}

The proof of the barber paradox annotated with the Lean tactic state after each tactic/step is provided below. This is also provided at this path in the code repository: \texttt{examples/Barber.lean}\todo{check filepath}. 

\begin{lean*}
theorem no_barber : ¬ existsBarber := by
  /-
  ⊢ ¬existsBarber
  -/
  simp [existsBarber, shavesThemselves]
  /-
  ⊢ ∀ (x : Person), ∃ x_1, ¬(¬x_1 = shaved_by x_1 ↔ shaved_by x_1 = x)
  -/
  intro b
  /-
  b : Person
  ⊢ ∃ x, ¬(¬x = shaved_by x ↔ shaved_by x = b)
  -/
  existsi b
  /-
  b : Person
  ⊢ ¬(¬b = shaved_by b ↔ shaved_by b = b)
  -/
  tauto
  done    
\end{lean*}

\subsection[]{Mutual Exclusion Protocol Specification \& Proofs}\label{appendix:mutex-proof}
The Forge specification and Lean proofs for the mutual exclusion protocol described in \cref{sec:mutex} are provided at this path in the code repository: \texttt{examples/Mutex.lean}. The specification spans lines 14--85 and proofs span lines 89--336. 

The Forge specification for this example is taken from \cite{l4s} with slight modifications for more than two processes. We contribute the entirety of the Lean proof of correctness of this protocol. 

\subsection[]{Additional Examples}\label{appendix:additional-examples}
Additional examples are also provided in the \texttt{examples} directory. 

Two examples, Tic-Tac-Toe and `Grandpa', are based on course content from Logic for Systems \cite{l4s} with minimal modifications. They serve solely to illustrate the capabilities of \textsc{Lforge} on translating existing programs, and we do not claim to make additional contributions to these models. Tic-Tac-Toe is at the following path: \texttt{examples/Board.lean}, and the `Grandpa' is at the following path: \texttt{examples/Grandpa.lean}. 

The specification regarding the two-phase atomic commitment protocol is inspired by \cite{distributedforge} but does not use any code directly from the repository. This serves as an example of a more complex distributed system protocol that is translatable. 